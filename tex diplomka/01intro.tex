\chapter{Úvod}\label{chap:intro}

V dnešnej dobe existuje veľké množstvo GPS navigácii pre automobily. Male množstvo
aplikácii je určené pre Bratislavu z integráciou hromadnej dopravy. Podobne riešenia ma aj
google maps api to je nefunkčné. Nakoľko najpoužívanejšia navigácia je google maps ta
v tejto lokalite umožňuje iba využitie automobilovej dopravy cyklistickej dopravy a peši
presun. Cieľom tejto diplomovej prace je Preskúmať možnosti a technické prostriedky
navigácie osôb prostredníctvom webovej aplikácie v mobilnom zariadení s GPS.
Aplikácia bude zbierať dáta zúčastnených osôb, na základe ktorých navrhne časovo i
finančne najvýhodnejšiu trasu z bodu A do bodu B.
Predpokladané prostriedky dopravy sú:

\begin{itemize}
	\item peší presun
	\item MHD.
\end{itemize}
Aplikácia bude osobu navigovať, monitorovať a robiť prípadné zmeny trasy.
Uložené trasy užívateľov bude potrebné vhodnými metódami (napr. Kalmanov filter)
previesť do konzistentnej podoby vzhľadom na použitú mapu mesta.
V rámci trasy sa bude automaticky detekovať spôsob presunu osoby (pešo, linka MHD)